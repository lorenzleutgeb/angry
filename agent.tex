\section{The \ah agent}
\label{sec:agent}
% Describe the Angry Birds scenario/setting and how the agent works
\ah is an artificial player for Angry
Birds, which reasoning with respect to
the world knowledge is carried out by
means of computing the answer set of a
HEX-program.
It originated as a re-implementation of
the naive agent provided by the organizers
of the Angry Birds AI Competition~\cite{angryAI},
with general improvements on main components
of it and a complete rewriting of the
\emph{planning} component as a collection
of HEX-programs.

In this section, an overview of the \ah
agent will be presented, with a focus on
the declarative part and its features. 
For a full and detailed explanation of
the implementing process, of the agent
layout and additional considerations on
performance, the reader can refer to~\cite{angryhex}.

\subsection{Base Framework}

The Base Framework, upon which \ah is built,
consists of several modules, dealing with
different aspects of the game, from 
the interaction with the game environment,
provided by the \emph{Angry Birds} browser
extension and the \emph{Proxy} component of
the \emph{Game Server}, to the communication
between the agent client and the server,
mediated through the server/client \emph{Communication Port};
the following list describes the parts of main interest:
\begin{description}
    \item[Vision] This module segments the images
    captured from the game environment, returning
    the minimum bounding rectangles of essential
    objects, as well as relevant information,
    like the types of birds available, the material
    of which bricks are composed and where pigs are placed.
    \item[Trajectory] This component estimates
    the trajectory followed by a bird after being
    released; here, orientation and distance of 
    the release point with respect to the slingshot
    are taken into account.
    \item[Planning] also called \emph{AI agent} in the
    original setting, this part delivers the order
    of the played levels, the choice of birds to use
    and other strategy-related choices.
\end{description}

\subsection{Interaction with HEX-programs}

The main contribution brought by \ah is
found in the planning component, which has
been rewritten as a collection of HEX-programs.
These are partitioned in two layers, namely
the \emph{Tactic} and the \emph{Strategy} layer.
Then, these programs are feeded to the
\textsc{dlv-hex} solver~\cite{dlvHEX},
which returns answer sets containing
information about the next moves in the game.