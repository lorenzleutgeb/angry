\documentclass[smaller, dvipsnames]{beamer}

\usetheme[numbering=fraction,%
          block=fill]{metropolis} % Use metropolis theme
\setbeamercovered{transparent}

\usepackage[utf8]{inputenc}
\usepackage[T1]{fontenc}

\usepackage{xspace}

\newcommand{\ah}{Angry-HEX\xspace}
\newcommand{\ab}{Angry Birds\xspace}
\newcommand{\abc}{Angry Birds AI Competition\xspace}
\newcommand{\al}{Alpha\xspace}

\title{Complex Predicates vs Complex Objects}
\subtitle{A case study on ASP for Implementing Artificial Agents}
\author{Filippo De Bortoli \and Lorenz Leutgeb \and Cosimo Persia}
\institute{European Master's Program in Computational Logic, TU Dresden}
\date{February 16th, 2018}

\begin{document}

\maketitle

\begin{frame}
    \frametitle{Outline}
\end{frame}

\begin{frame}
 	\frametitle{Angry Birds - The Game}
	\begin{figure}
  		\includegraphics[width=300pt]{./img/angry-birds.jpg}
	\end{figure}
	\begin{itemize}
		\item<1-> Eliminate all pigs
		\item<2-> Reach high score
	\end{itemize}
\end{frame}

\begin{frame}
 	\frametitle{Angry Birds - Challanges for AI}
		\begin{itemize}
    		\item<1-> Physics
    		{\par\centering\includegraphics<2>[width=5cm]{./img/birds.png}\par}
    		\item<3-> Planning
    		{\par\centering\includegraphics<4>[width=5cm]{./img/planning.png}\par}
    		\item<5-> Computer Vision
    		{\par\centering\includegraphics<6>[width=5cm]{./img/vision.jpg}\par}
    		\item<7-> Knowledge Representation
  		\end{itemize}
\end{frame}

\section{ASP and HEX-programs}
\begin{frame}
 	\begin{center}
 	\begin{itemize}
	\item<1->[] Declarative Programming ALGORITM = LOGIC + CONTROL
		\begin{align*}
			&append ([\:], X, X). \\
			&append ([X|Y], Z, [X|T ]) \leftarrow append (Y, Z, T ). \\
			&reverse([ ], [ ]).\\
			&reverse([X|Y ], Z) \leftarrow append (U, [X], Z), reverse(Y, U ).
		\end{align*}
  \item<2>[] Order of clauses and subgoals does not matter.
  		\begin{align*}
			reverse([X|Y ], Z) \leftarrow reverse(Y, U ), append (U, [X], Z).
		\end{align*}
	\end{itemize}	
	\end{center}
 
\end{frame}

\begin{frame}
    \frametitle{Stable model semantics}
    \begin{center}
    	\begin{align*}
			&pig((88,34)). \\
			&easy\_taget(X) \leftarrow pig(X), not\: difficult\_taget(X). \\ 
			&difficult\_taget(X) \leftarrow pig(X), not\: easy\_taget(X) 
		\end{align*}
    \end{center}
    \begin{itemize}
    	\item<2->[] models which reflects natural intuition
    	\item<3->[]
    		\begin{align*}
				&\only<4>{\textcolor{red}}{ M_1= \{pig((88,34)), easy\_taget((88,34))\}  } \\
				&\only<4>{\textcolor{red}}{ M_2= \{pig((88,34)), difficult\_taget((88,34))\}  }\\
				&M_3= \{pig((88,34)), easy\_taget((88,34)), difficult\_taget((88,34))\} \\
				&M_4= \{pig((88,34))\} \\
    		\end{align*}
    \end{itemize}
\end{frame}
%maybe another example of stable model here

\begin{frame}
    \frametitle{External atoms}
    \begin{itemize}
    	\item<1-> allows bidirectional communication with external sources
    	\item<2->[] \[ \&g[q_1,\dots,q_k](t_1,\dots,t_l) \]
    	\item<3-> Example: we want to use an arbitrary computable function that access an rdf file from the web
    	\item<4->[] \[ \&rdf[url](X,Y,Z) \]
    	\item<5->[] 
    		\begin{align*}
				bel(X,Y,Z) \leftarrow \&rdf[url](X,Y,Z)
    		\end{align*}
    \end{itemize}
\end{frame}

\begin{frame}
    \frametitle{HEX-programs}
    \begin{itemize}
    	\item<1-> Generalization of disjunctive extended logic programs under the answer set semantics.
    	\item<2-> Rules
    	\item<2->[] \[ a_1, \lor \dots \lor a_n \leftarrow b_1, \dots , b_m, \neg b_m+1, \dots, b_n \; (m,n \geq 0)\]
    	\item<3-> External atoms
    	\item<3->[] \[ \&g[q_1,\dots,q_k](t_1,\dots,t_l) \]
    \end{itemize}
\end{frame}

\section{\ah}

\begin{frame}
    \frametitle{Base Framework}
\end{frame}

\begin{frame}
    \frametitle{Interaction with HEX-programs}
\end{frame}

\section{From complex predicates to complex objects}

\begin{frame}
    C
\end{frame}

\begin{frame}[standout]
    Thank you!
\end{frame}

\end{document}